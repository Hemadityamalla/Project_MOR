  % --------------------------------------------------------------
% This is all preamble stuff that you don't have to worry about.
% Head down to where it says "Start here"
% --------------------------------------------------------------
 
\documentclass[12pt]{article}
 
\usepackage[margin=1in]{geometry} 
\usepackage{amsmath,amsthm,amssymb}
\usepackage{subcaption}
\usepackage{graphicx}
\usepackage{physics}
\usepackage{listings}
\usepackage{hyperref}
\usepackage{mathtools}
\usepackage{float}
\hypersetup{
    colorlinks=true,
    linkcolor=blue,
    filecolor=magenta,      
    urlcolor=cyan,
}
 
\urlstyle{same}

%Setup for adding MATLAB code
\usepackage{listings}
\usepackage{color} %red, green, blue, yellow, cyan, magenta, black, white
\definecolor{mygreen}{RGB}{28,172,0} % color values Red, Green, Blue
\definecolor{mylilas}{RGB}{170,55,241}
\definecolor{mGreen}{rgb}{0,0.6,0}
\definecolor{mGray}{rgb}{0.5,0.5,0.5}
\definecolor{mPurple}{rgb}{0.58,0,0.82}
\definecolor{backgroundColour}{rgb}{0.95,0.95,0.92}

\lstdefinestyle{matlab}{
    backgroundcolor=\color{backgroundColour},   
    commentstyle=\color{mGreen},
    keywordstyle=\color{magenta},
    numberstyle=\tiny\color{mGray},
    stringstyle=\color{mPurple},
    basicstyle=\footnotesize,
    breakatwhitespace=false,         
    breaklines=true,                 
    captionpos=b,                    
    keepspaces=true,                 
    numbers=left,                    
    numbersep=5pt,                  
    showspaces=false,                
    showstringspaces=false,
    showtabs=false,                  
    tabsize=2,
    language=MATLAB
} 
%Packages to add Pseudocode algorithms
\usepackage{algorithm}
\usepackage{algorithmicx}
\usepackage{algpseudocode}

%Custom stuff used to add comments
\newcommand{\TODO}[1]{{\color{red} TODO: #1}}
\newcommand{\XZ}[1]{{\color{blue} XZ: #1}}
\newcommand{\HM}[1]{{\color{green} HM: #1}}

\newcommand{\delx}{\Delta x}
\newcommand{\delt}{\Delta t}
\newcommand{\bu}{\mathbf{u}}
\newcommand{\bA}{\mathbf{A}}
\newcommand{\bfn}{\mathbf{f}}


 
\begin{document}
 
% --------------------------------------------------------------
%                         Start here
% --------------------------------------------------------------
 
%\renewcommand{\qedsymbol}{\filledbox}
\title{%
  Model Order Reduction Project \\
  \large Heat Diffusion} 
 
 \author{ %replace with your name
Xinyu Zeng, 1301462 \\
Hemaditya Malla, 1282484
}
 
\maketitle
\tableofcontents
\pagebreak

\section*{Problem 1}
\addcontentsline{toc}{section}{\protect\numberline{}Problem 3}

This system is non-linear and time-invariant.\\
According to description, this model is isotropic.\\ Therefor,
\begin{equation}
\rho (x,y)c(x,y)\frac {\partial T} {\partial t}(x,y,t)=
\begin{bmatrix}
\frac {\partial} {\partial x} & \frac {\partial} {\partial y}
\end{bmatrix}
\begin{bmatrix}
\kappa(x,y) & 0\\
0 & \kappa(x,y) 
\end{bmatrix}
\begin{bmatrix}
\frac {\partial T} {\partial x}(x,y,t)\\
\frac {\partial T} {\partial y}(x,y,t)
\end{bmatrix}
+u(x,y,t)
\end{equation}\\
i.e.
\begin{equation}
\rho (x,y)c(x,y)\frac {\partial T} {\partial t}(x,y,t)=
(\frac {\partial \kappa} {\partial x}(x,y) \frac {\partial T} {\partial x}(x,y,t)+\frac {\partial \kappa} {\partial y}(x,y) \frac {\partial T} {\partial y}(x,y,t))+u(x,y,t)
\end{equation}

\subsection*{non-homogeneous} 
\subsubsection*{Non-linear}
Firstly, just consider the left side of the equation(2),
\begin{equation}
\rho(x_1+x_2,y_1+y_2) c(x_1+x_2,y_1+y_2) \frac {\partial T} {\partial t}(x_1+x_2,y_1+y_2,t)
\end{equation}\\
Since it's unknown whether $\rho$ and c is linear(because the part of T is only associated with t, this part could be ignored), even though they are both linear,like equation (5), it is still non-linear.

\begin{equation}
\small{\rho(x_1+x_2,y_1+y_2) c(x_1+x_2,y_1+y_2) =(\rho(x_1,y_1)+\rho(x_2,y_2)) (c(x_1,y_1)+c(x_2,y_2))}
\end{equation}

\begin{equation}
\scriptsize{\rho(x_1,y_1)c(x_1,y_1)+\rho(x_1,y_1)c(x_2,y_2)+\rho(x_2,y_2)c(x_1,y_1)+\rho(x_2,y_2))c(x_2,y_2) \neq (\rho(x_1,y_1)c(x_1,y_1)+\rho(x_2,y_2))c(x_2,y_2))}
\end{equation}

\subsubsection*{time-invariant}
input delayed: $u_d(t)=u(t+\delta)$\\
\begin{equation}
\rho (x,y)c(x,y)\frac {\partial T} {\partial t}(x,y,t)-(\frac {\partial \kappa} {\partial x}(x,y) \frac {\partial T} {\partial x}(x,y,t)+\frac {\partial \kappa} {\partial y}(x,y) \frac {\partial T} {\partial y}(x,y,t))=u(x,y,t+\delta)
\end{equation}\\
output delayed:$T_d(t)=T(t+\delta)$\\
\begin{equation}
\rho (x,y)c(x,y)\frac {\partial T} {\partial t}(x,y,t+\delta)-(\frac {\partial \kappa} {\partial x}(x,y) \frac {\partial T} {\partial x}(x,y,t+\delta)+\frac {\partial \kappa} {\partial y}(x,y) \frac {\partial T} {\partial y}(x,y,t+\delta))=u(x,y,t+\delta)
\end{equation}\\
The right side of equation(6)and (7) is equal, so it is time-invariant.

\subsection*{homogeneous}
\subsubsection*{Non-linear}
When $\rho$ and c is constant, consider the right side of equation(2),\\
\begin{equation}
\frac {\partial \kappa} {\partial x}(x_1+x_2,y_1+y_2) \frac {\partial T} {\partial x}(x_1+x_2,y_1+y_2,t)+\frac {\partial \kappa} {\partial y}(x_1+x_2,y_1+y_2) \frac {\partial T} {\partial y}(x_1+x_2,y_1+y_2,t)+u(x_1+x_2,y_1+y_2,t)
\end{equation}\\

since $\kappa$ is coupled with T, the first item\\
\begin{equation}
\frac {\partial \kappa} {\partial x}(x_1+x_2) \frac {\partial T} {\partial x}(x_1+x_2) \neq \frac {\partial \kappa} {\partial x}(x_1)\frac {\partial T} {\partial x}(x_1)+\frac {\partial \kappa} {\partial x}(x_2)\frac {\partial T} {\partial x}(x_2)
\end{equation}

\subsubsection*{time-invariant}
Because only u and T is associated with time that no matter the system is homogeneous or not, it is time-invariant. (Proof as equation(6) and (7))


\section*{Problem 2}
\addcontentsline{toc}{section}{\protect\numberline{}Problem 4}
%\begin{figure}[H]
%\begin{subfigure}{0.48\textwidth}
%\includegraphics[width=\linewidth]{22_21_ukappa_contour.eps}
%\end{subfigure}\hspace*{\fill}
%\begin{subfigure}{0.48\textwidth}
%\includegraphics[width=\linewidth]{22_21_ukappa_3d.eps}
%\end{subfigure}
%
%\medskip
%\begin{subfigure}{0.48\textwidth}
%\includegraphics[width=\linewidth]{42_41_ukappa_contour.eps}
%\end{subfigure}\hspace*{\fill}
%\begin{subfigure}{0.48\textwidth}
%\includegraphics[width=\linewidth]{42_41_ukappa_3d.eps}
%\end{subfigure}
%
%\medskip
%\begin{subfigure}{0.48\textwidth}
%\includegraphics[width=\linewidth]{82_81_ukappa_contour.eps}
%\end{subfigure}\hspace*{\fill}
%\begin{subfigure}{0.48\textwidth}
%\includegraphics[width=\linewidth]{82_81_ukappa_3d.eps}
%\end{subfigure}
%
%\caption{Comparison of the numerical (left) and exact (right) solutions for the 2D convection equation using the \textbf{unlimited $\kappa=1/3$ discretization in space}. The contours shown are for values of $c = 10^{-6} + n, \ n=0,\ldots, 9$.}
%\end{figure}

\section*{Problem 3}
\addcontentsline{toc}{section}{\protect\numberline{}Problem 10}

\section*{Problem 4}
\addcontentsline{toc}{section}{\protect\numberline{}Problem 11}
\bibliography{references} 
\bibliographystyle{ieeetr}
\pagebreak
\appendix
\end{document}
