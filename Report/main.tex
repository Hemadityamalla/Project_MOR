  % --------------------------------------------------------------
% This is all preamble stuff that you don't have to worry about.
% --------------------------------------------------------------
 
\documentclass[12pt]{article}
 
\usepackage[margin=1in]{geometry} 
\usepackage{amsmath,amsthm,amssymb}
\usepackage{subcaption}
\usepackage{graphicx}
\usepackage{physics}
\usepackage{listings}
\usepackage{hyperref}
\usepackage{mathtools}
\usepackage{float}
\hypersetup{
    colorlinks=true,
    linkcolor=blue,
    filecolor=magenta,      
    urlcolor=cyan,
}
 
\urlstyle{same}

%Setup for adding MATLAB code
\usepackage{listings}
\usepackage{color} %red, green, blue, yellow, cyan, magenta, black, white
\definecolor{mygreen}{RGB}{28,172,0} % color values Red, Green, Blue
\definecolor{mylilas}{RGB}{170,55,241}
\definecolor{mGreen}{rgb}{0,0.6,0}
\definecolor{mGray}{rgb}{0.5,0.5,0.5}
\definecolor{mPurple}{rgb}{0.58,0,0.82}
\definecolor{backgroundColour}{rgb}{0.95,0.95,0.92}

\lstdefinestyle{matlab}{
    backgroundcolor=\color{backgroundColour},   
    commentstyle=\color{mGreen},
    keywordstyle=\color{magenta},
    numberstyle=\tiny\color{mGray},
    stringstyle=\color{mPurple},
    basicstyle=\footnotesize,
    breakatwhitespace=false,         
    breaklines=true,                 
    captionpos=b,                    
    keepspaces=true,                 
    numbers=left,                    
    numbersep=5pt,                  
    showspaces=false,                
    showstringspaces=false,
    showtabs=false,                  
    tabsize=2,
    language=MATLAB
} 
%Packages to add Pseudocode algorithms
\usepackage{algorithm}
\usepackage{algorithmicx}
\usepackage{algpseudocode}

%Custom stuff used to add comments
\newcommand{\TODO}[1]{{\color{red} TODO: #1}}
\newcommand{\XZ}[1]{{\color{blue} XZ: #1}}
\newcommand{\HM}[1]{{\color{green} HM: #1}}

\newcommand{\delx}{\Delta x}
\newcommand{\delt}{\Delta t}
\newcommand{\bu}{\mathbf{u}}
\newcommand{\bA}{\mathbf{A}}
\newcommand{\bfn}{\mathbf{f}}
%Use these commands instead of having to type out all of the derivative symbols in a fraction
\newcommand{\parder}[2]{\frac{\partial #1}{\partial #2}} 
\newcommand{\der}[2]{\frac{\mathrm{d} #1}{\mathrm{d} #2}}
%Use these commands while using norms and innerproducts
\let\norm\undefined %This line removes the definition of norm in the amsmath package since I have my own definition below
\newcommand{\norm}[1]{\left\lVert #1 \right\rVert}
\newcommand{\dotprod}[2]{\left\langle #1, #2 \right\rangle}


 
\begin{document}
 
% --------------------------------------------------------------
%                         Start here
% --------------------------------------------------------------
 
%\renewcommand{\qedsymbol}{\filledbox}
\title{%
  Model Order Reduction Project \\
  \large Heat Diffusion} 
 
 \author{ %replace with your name
Xinyu Zeng, 1301462 \\
Hemaditya Malla, 1282484
}
 
\maketitle
\tableofcontents
\pagebreak
\section*{Introduction}
\TODO{Write the intro later- define the physical parameters, domain}

\begin{equation}
 \begin{cases}
    \rho(x,y)c(x,y)\parder{T}{t}(x,y,t) = \begin{bmatrix}\parder{ }{x} & \parder{ }{y}\end{bmatrix}K(x,y)\begin{bmatrix}\parder{T(x,y,t)}{x} \\ \parder{T(x,y,t)}{y}\end{bmatrix} + u(x,y,t) \quad (0, T] \times \Omega, \\
    \parder{T(0,y,t)}{x} = \parder{L_x,y,t}{x} = 0, \quad y\in[0,L_y] \times [0,T],   \\
    \parder{T(x,0,t)}{y} = \parder{x,L_y,t}{y} = 0, \quad x\in[0,L_x] \times [0,T],   \\
    T(x,y,0) = T_0(x,y), \quad \{t = 0\} \times \Omega,
\end{cases}
\end{equation}
where $\Omega := [0,L_x] \times [0,L_y]$ and $T_0(x,y)$ is a physically realistic initial temperature profile.
\section*{Problem 1}
\addcontentsline{toc}{section}{\protect\numberline{}Problem 1}

This system is non-linear and time-invariant.\\
According to description, this model is isotropic.\\ Therefore,
\begin{equation}
\rho (x,y)c(x,y)\parder{T}{t}(x,y,t)=
\begin{bmatrix}
\parder{ }{x} & \parder{ }{y}
\end{bmatrix}
\begin{bmatrix}
\kappa(x,y) & 0\\
0 & \kappa(x,y) 
\end{bmatrix}
\begin{bmatrix}
\parder{T}{x}(x,y,t)\\
\parder{T}{y}(x,y,t)
\end{bmatrix}
+u(x,y,t),
\label{eqn:main_prob1}
\end{equation}
i.e.,
\begin{equation}
\rho (x,y)c(x,y)\parder{T}{t}(x,y,t)= \parder{ }{x}\bigg(\kappa(x,y)\parder{T}{x} (x,y,t)\bigg) + \parder{ }{y}\bigg(\kappa(x,y)\parder{T}{y} (x,y,t)\bigg)+u(x,y,t).
\label{eqn:simplified_prob1}
\end{equation}
\textbf{Remark:} Here onwards(\HM{Is that correct grammar?}), it is implicitly implied that $T = T(x,y,t)$ unless specified otherwise and $(x,y,t)$ is dropped for notational convenience.
\subsection*{Non-homogeneous} 
\subsubsection*{Nonlinearity}
In order to check for the linearity of \eqref{eqn:simplified_prob1} the following solution is assumed:
\begin{equation}
T(x,y,t) = T_1(x,y,t) + T_2(x,y,t),
\label{eqn:prob1_soln}
\end{equation}
where $T_1(x,y,t) \ \& \ T_2(x,y,t)$ are solutions of the PDE \eqref{eqn:simplified_prob1}.
Substituting the solution \eqref{eqn:prob1_soln} into the left hand side of the PDE \eqref{eqn:simplified_prob1}, we get,
\begin{equation}
\rho(x,y)c(x,y) \parder{T_1 + T_2}{t} (x,y,t) = \rho
\end{equation}
%\begin{equation}
%\rho(x_1+x_2,y_1+y_2) c(x_1+x_2,y_1+y_2) \frac {\partial T} {\partial t}(x_1+x_2,y_1+y_2,t)
%\end{equation}\\
%Since it's unknown whether $\rho$ and c is linear(because the part of T is only associated with t, this part could be ignored), even though they are both linear,like equation (5), it is still non-linear.
%
%\begin{equation}
%\small{\rho(x_1+x_2,y_1+y_2) c(x_1+x_2,y_1+y_2) =(\rho(x_1,y_1)+\rho(x_2,y_2)) (c(x_1,y_1)+c(x_2,y_2))}
%\end{equation}
%
%\begin{equation}
%\scriptsize{\rho(x_1,y_1)c(x_1,y_1)+\rho(x_1,y_1)c(x_2,y_2)+\rho(x_2,y_2)c(x_1,y_1)+\rho(x_2,y_2))c(x_2,y_2) \neq (\rho(x_1,y_1)c(x_1,y_1)+\rho(x_2,y_2))c(x_2,y_2))}
%\end{equation}

\subsubsection*{Time-invariant}
input delayed: $u_d(t)=u(t+\delta)$\\
\begin{equation}
\rho (x,y)c(x,y)\frac {\partial T} {\partial t}(x,y,t)-(\frac {\partial \kappa} {\partial x}(x,y) \frac {\partial T} {\partial x}(x,y,t)+\frac {\partial \kappa} {\partial y}(x,y) \frac {\partial T} {\partial y}(x,y,t))=u(x,y,t+\delta)
\end{equation}\\
output delayed:$T_d(t)=T(t+\delta)$\\
\begin{equation}
\rho (x,y)c(x,y)\frac {\partial T} {\partial t}(x,y,t+\delta)-(\frac {\partial \kappa} {\partial x}(x,y) \frac {\partial T} {\partial x}(x,y,t+\delta)+\frac {\partial \kappa} {\partial y}(x,y) \frac {\partial T} {\partial y}(x,y,t+\delta))=u(x,y,t+\delta)
\end{equation}\\
The right side of equation(6)and (7) is equal, so it is time-invariant.

\subsection*{homogeneous}
\subsubsection*{Non-linear}
When $\rho$ and c is constant, consider the right side of equation(2),\\
\begin{equation}
\frac {\partial \kappa} {\partial x}(x_1+x_2,y_1+y_2) \frac {\partial T} {\partial x}(x_1+x_2,y_1+y_2,t)+\frac {\partial \kappa} {\partial y}(x_1+x_2,y_1+y_2) \frac {\partial T} {\partial y}(x_1+x_2,y_1+y_2,t)+u(x_1+x_2,y_1+y_2,t)
\end{equation}\\

since $\kappa$ is coupled with T, the first item\\
\begin{equation}
\frac {\partial \kappa} {\partial x}(x_1+x_2) \frac {\partial T} {\partial x}(x_1+x_2) \neq \frac {\partial \kappa} {\partial x}(x_1)\frac {\partial T} {\partial x}(x_1)+\frac {\partial \kappa} {\partial x}(x_2)\frac {\partial T} {\partial x}(x_2)
\end{equation}

\subsubsection*{time-invariant}
Because only u and T is associated with time that no matter the system is homogeneous or not, it is time-invariant. (Proof as equation(6) and (7))


\section*{Problem 2}
\addcontentsline{toc}{section}{\protect\numberline{}Problem 2}
Homogeneous model assumption implies the following:
\begin{itemize}
	\item $l_x = l_y = 0$.
	\item $\rho(x,y) = \rho$ $\kappa(x,y) = \kappa$ and $c(x,y) = c$, where $\rho, c, \kappa$ are positive constants.
\end{itemize}
Applying the above assumptions to the model gives
\begin{equation}
\rho c \parder{T}{t} = \kappa \bigg(\parder{^2 T}{x^2} + \parder{^2 T}{y^2}\bigg) + u(x,y,t),
\label{eqn:prob2_model}
\end{equation}
as the final equation. \\
Let the source term be zero, i.e., $u(x,y,t) = 0$. Consider the function 
\begin{equation}
T(x,y,t) = a(t)\psi(x)\phi(y),
\label{eqn:prob2_soln}
\end{equation}
where $a(t)$, $\psi(x)$ and $\phi(y)$ are real-valued functions on $\mathbb{R}$, $[0,L_x]$ and $[0,L_y]$ respectively. \\
Substituting \eqref{eqn:prob2_soln} into \eqref{eqn:prob2_model},
\begin{equation}
\rho c \parder{\big(a(t)\psi(x)\phi(y)\big)}{t} = \kappa \bigg(\parder{^2 \big(a(t)\psi(x)\phi(y)\big)}{x^2} + \parder{^2 \big(a(t)\psi(x)\phi(y)\big)}{y^2}\bigg).
\end{equation}
This gives,   
\begin{equation}
\rho c \psi(x) \phi(y) \der{a(t)}{t} = \kappa a(t)\bigg(\phi(y)\der{^2 \psi(x)}{x^2} + \psi(x)\der{^2 \phi(y)}{y^2}\bigg).
\end{equation}
Dividing throughout by \eqref{eqn:prob2_soln}, 
\begin{equation}
\frac{\rho c}{a(t)} \der{a(t)}{t} = \kappa \bigg(\frac{1}{\psi(x)}\der{^2 \psi(x)}{x^2} + \frac{1}{\phi(y)}\der{^2 \phi(y)}{y^2}\bigg).
\end{equation}
Rearranging, gives,
\begin{equation}
\frac{\rho c}{a(t)} \der{a(t)}{t} - \kappa \bigg(\frac{1}{\psi(x)}\der{^2 \psi(x)}{x^2} + \frac{1}{\phi(y)}\der{^2 \phi(y)}{y^2}\bigg) = 0.
\end{equation}
The three terms in the above equation are functions of the three independent variables $x, y, t$. So, in order for the above equation to be satisfied, each of the term must be constant. Consider a constant $\alpha > 0$. The three terms are then given as
\begin{equation}
\frac{\rho c}{a(t)} \der{a(t)}{t} = \alpha, \qquad \frac{1}{\psi(x)}\der{^2 \psi(x)}{x^2} =   \frac{1}{\phi(y)}\der{^2 \phi(y)}{y^2} = \frac{\alpha}{2 \kappa}.
\end{equation}
Upon simplifying, the following differential equations are obtained: 
\begin{subequations}
\begin{equation}
\der{a(t)}{t} - \lambda a(t) = 0,
\label{eqn:prob2_sub1}
\end{equation}
\begin{equation}
\der{^2 \psi(x)}{x^2} - \lambda_x \psi(x) = 0,
\label{eqn:prob2_sub2}
\end{equation}
\begin{equation}
\der{^2 \phi(y)}{y^2} - \lambda_y \phi(y) = 0,
\label{eqn:prob2_sub3}
\end{equation}
where $\lambda = \frac{\alpha}{\rho c}$, $\lambda_x  = \lambda_y = \frac{\alpha}{2 \kappa}$. \\
\end{subequations}
Thus it can be seen that for equation\eqref{eqn:prob2_soln} to be a solution to the PDE \eqref{eqn:prob2_model} when $u(x,y,t) = 0$, $a(t)$, $\psi(x)$ and $\phi(y)$ need to satisfy the equations \eqref{eqn:prob2_sub1}, \eqref{eqn:prob2_sub2} and \eqref{eqn:prob2_sub3} respectively.
%\begin{figure}[H]
%\begin{subfigure}{0.48\textwidth}
%\includegraphics[width=\linewidth]{22_21_ukappa_contour.eps}
%\end{subfigure}\hspace*{\fill}
%\begin{subfigure}{0.48\textwidth}
%\includegraphics[width=\linewidth]{22_21_ukappa_3d.eps}
%\end{subfigure}
%
%\medskip
%\begin{subfigure}{0.48\textwidth}
%\includegraphics[width=\linewidth]{42_41_ukappa_contour.eps}
%\end{subfigure}\hspace*{\fill}
%\begin{subfigure}{0.48\textwidth}
%\includegraphics[width=\linewidth]{42_41_ukappa_3d.eps}
%\end{subfigure}
%
%\medskip
%\begin{subfigure}{0.48\textwidth}
%\includegraphics[width=\linewidth]{82_81_ukappa_contour.eps}
%\end{subfigure}\hspace*{\fill}
%\begin{subfigure}{0.48\textwidth}
%\includegraphics[width=\linewidth]{82_81_ukappa_3d.eps}
%\end{subfigure}
%
%\caption{Comparison of the numerical (left) and exact (right) solutions for the 2D convection equation using the \textbf{unlimited $\kappa=1/3$ discretization in space}. The contours shown are for values of $c = 10^{-6} + n, \ n=0,\ldots, 9$.}
%\end{figure}

\section*{Problem 3}
\addcontentsline{toc}{section}{\protect\numberline{}Problem 3}
The space of square-integrable functions in $\Omega$ is defined as 
\begin{equation}
L_2(\Omega) = \left\{ f(x,y):\Omega\rightarrow\mathbb{R} \ \bigg| \ \int_0^{L_y}\int_0^{L_x} f(x,y)^2 \mathrm{d}x\mathrm{d}y < \infty \right\}
\label{eqn:l2space}
\end{equation}
The inner product for $L_2(\Omega)$ is defined as follows:
\begin{equation}
\dotprod{f}{g} := \int_0^{L_y}\int_0^{L_x} f(x,y)g(x,y) \mathrm{d}x\mathrm{d}y, \qquad f,g \in L_2(\Omega).
\label{eqn:l2_innerprod}
\end{equation}
The norm associated with $L_2(\Omega)$ is defined as follows:
\begin{equation}
\norm{f}_{L_2} = \dotprod{f}{f}^{\frac{1}{2}}, \qquad f \in L_2(\Omega).
\label{eqn:l2_norm}
\end{equation}
\section*{Problem 4}
\addcontentsline{toc}{section}{\protect\numberline{}Problem 4}
The spectral expansion of $T(x,y,t)$ is given as
\begin{equation}
T(x,y,t) = \sum_{k=0}^\infty\sum_{l=0}^\infty a_{kl}(t)\psi_k(x)\phi_l(y).
\end{equation}
Taking $u(x,y,t) = 0$ and substituting the above solution in equation \eqref{eqn:prob2_model}, the following is obtained:
\begin{equation}
\rho c\ parder{ }{t} \left(\sum_{k=0}^\infty\sum_{l=0}^\infty a_{k,l}(t)\psi_k(x)\phi_l(y)\right) = \kappa \left(\parder{^2}{x^2} + \parder{^2}{y^2}\right)\sum_{k=0}^\infty\sum_{l=0}^\infty a_{kl}(t)\psi_k(x)\phi_l(y).
\end{equation}
The partial derivatives can be written as regular(\HM{?}) derivatives as they are now operating on functions of a single variable.
\begin{equation}
\Rightarrow \rho c \sum_{k=0}^\infty\sum_{l=0}^\infty \der{^2 a_{k,l}(t)}{t} \psi_k(x)\phi_l(y) = \kappa \sum_{k=0}^\infty\sum_{l=0}^\infty a_{k,l}(t) \left(\der{^2 \psi_{k}(x)}{x^2}\phi_l(y) + \psi_k(x)\der{^2\phi_l(y)}{y^2}\right)
\end{equation}
\textbf{Remark:} It has been made clear that $\psi = \psi(x)$ and $\phi = \phi(y)$, so $(x) \& (y)$ are dropped henceforth.\\
The function $\psi_i(x)\phi_j(y)$ is multiplied throughout and the equation is integrated over the whole domain $\Omega$ as follows:
\begin{multline}
\rho c \int_{0}^{L_y}\int_0^{L_x}\sum_{k=0}^\infty\sum_{l=0}^\infty \der{^2 a_{k,l}(t)}{t} \psi_k(x)\phi_l(y) \psi_i(x)\phi_j(y) = \\ \kappa \int_{0}^{L_y}\int_0^{L_x} \sum_{k=0}^\infty\sum_{l=0}^\infty a_{k,l}(t) \left(\der{^2 \psi_{k}(x)}{x^2}\phi_l(y)\psi_i(x)\phi_j(y) + \psi_k(x)\der{^2\phi_l(y)}{y^2}\psi_i(x)\phi_j(y)\right).
\end{multline}
Re-writing the above equation using the inner product notation \eqref{eqn:l2_innerprod}, we get, 
\begin{multline}
\rho c \sum_{k=0}^\infty\sum_{l=0}^\infty \der{^2 a_{k,l}(t)}{t} \dotprod{\psi_k(x)\phi_l(y)}{\psi_i(x)\phi_j(y)} = \\ \kappa \sum_{k=0}^\infty\sum_{l=0}^\infty a_{k,l}(t) \left(\dotprod{\der{^2 \psi_{k}(x)}{x^2}\phi_l(y)}{\psi_i(x)\phi_j(y)} + \dotprod{\psi_k(x)\der{^2\phi_l(y)}{y^2}}{\psi_i(x)\phi_j(y)}\right).
\end{multline}
\bibliography{references} 
\bibliographystyle{ieeetr}
\pagebreak
\appendix
\end{document}
