  % --------------------------------------------------------------
% This is all preamble stuff that you don't have to worry about.
% Head down to where it says "Start here"
% --------------------------------------------------------------
 
\documentclass[12pt]{article}
 
\usepackage[margin=1in]{geometry} 
\usepackage{amsmath,amsthm,amssymb}
\usepackage{subcaption}
\usepackage{graphicx}
\usepackage{physics}
\usepackage{listings}
\usepackage{hyperref}
\usepackage{mathtools}
\usepackage{float}
\hypersetup{
    colorlinks=true,
    linkcolor=blue,
    filecolor=magenta,      
    urlcolor=cyan,
}
 
\urlstyle{same}

%Setup for adding MATLAB code
\usepackage{listings}
\usepackage{color} %red, green, blue, yellow, cyan, magenta, black, white
\definecolor{mygreen}{RGB}{28,172,0} % color values Red, Green, Blue
\definecolor{mylilas}{RGB}{170,55,241}
\definecolor{mGreen}{rgb}{0,0.6,0}
\definecolor{mGray}{rgb}{0.5,0.5,0.5}
\definecolor{mPurple}{rgb}{0.58,0,0.82}
\definecolor{backgroundColour}{rgb}{0.95,0.95,0.92}

\lstdefinestyle{matlab}{
    backgroundcolor=\color{backgroundColour},   
    commentstyle=\color{mGreen},
    keywordstyle=\color{magenta},
    numberstyle=\tiny\color{mGray},
    stringstyle=\color{mPurple},
    basicstyle=\footnotesize,
    breakatwhitespace=false,         
    breaklines=true,                 
    captionpos=b,                    
    keepspaces=true,                 
    numbers=left,                    
    numbersep=5pt,                  
    showspaces=false,                
    showstringspaces=false,
    showtabs=false,                  
    tabsize=2,
    language=MATLAB
} 
%Packages to add Pseudocode algorithms
\usepackage{algorithm}
\usepackage{algorithmicx}
\usepackage{algpseudocode}

%Custom stuff used to add comments
\newcommand{\TODO}[1]{{\color{red} TODO: #1}}
\newcommand{\XZ}[1]{{\color{blue} XZ: #1}}
\newcommand{\HM}[1]{{\color{green} HM: #1}}

\newcommand{\delx}{\Delta x}
\newcommand{\delt}{\Delta t}
\newcommand{\bu}{\mathbf{u}}
\newcommand{\bA}{\mathbf{A}}
\newcommand{\bfn}{\mathbf{f}}
\newcommand{\parder}[2]{\frac{\partial #1}{\partial #2}}
\newcommand{\der}[2]{\frac{\mathrm{d} #1}{\mathrm{d} #2}}


 
\begin{document}
 
% --------------------------------------------------------------
%                         Start here
% --------------------------------------------------------------
 
%\renewcommand{\qedsymbol}{\filledbox}
\title{%
  Model Order Reduction Project \\
  \large Heat Diffusion} 
 
 \author{ %replace with your name
Xinyu Zeng, 123445 \\
Hemaditya Malla, 1282484
}
 
\maketitle
\tableofcontents
\pagebreak

\section*{Problem 1}
\addcontentsline{toc}{section}{\protect\numberline{}Problem 1}

\section*{Problem 2}
\addcontentsline{toc}{section}{\protect\numberline{}Problem 2}
Homogeneous model assumption implies the following:
\begin{itemize}
	\item $l_x = l_y = 0$.
	\item $\rho(x,y) = \rho$ $\kappa(x,y) = \kappa$ and $c(x,y) = c$, where $\rho, c, \kappa$ are positive constants.
\end{itemize}
Applying the above assumptions to the model gives
\begin{equation}
\rho c \parder{T}{t} = \kappa \bigg(\parder{^2 T}{x^2} + \parder{^2 T}{y^2}\bigg) + u(x,y,t),
\label{eqn:prob2_model}
\end{equation}
as the final equation. \\
Let the source term be zero, i.e., $u(x,y,t) = 0$. Consider the function 
\begin{equation}
T(x,y,t) = a(t)\psi(x)\phi(y),
\label{eqn:prob2_soln}
\end{equation}
where $a(t)$, $\psi(x)$ and $\phi(y)$ are real-valued functions on $\mathbb{R}$, $[0,L_x]$ and $[0,L_y]$ respectively. \\
Substituting \eqref{eqn:prob2_soln} into \eqref{eqn:prob2_model},
\begin{equation}
\rho c \parder{\big(a(t)\psi(x)\phi(y)\big)}{t} = \kappa \bigg(\parder{^2 \big(a(t)\psi(x)\phi(y)\big)}{x^2} + \parder{^2 \big(a(t)\psi(x)\phi(y)\big)}{y^2}\bigg).
\end{equation}
This gives,   
\begin{equation}
\rho c \psi(x) \phi(y) \der{a(t)}{t} = \kappa a(t)\bigg(\phi(y)\der{^2 \psi(x)}{x^2} + \psi(x)\der{^2 \phi(y)}{y^2}\bigg).
\end{equation}
Dividing throughout by \eqref{eqn:prob2_soln}, 
\begin{equation}
\frac{\rho c}{a(t)} \der{a(t)}{t} = \kappa \bigg(\frac{1}{\psi(x)}\der{^2 \psi(x)}{x^2} + \frac{1}{\phi(y)}\der{^2 \phi(y)}{y^2}\bigg).
\end{equation}
Rearranging, gives,
\begin{equation}
\frac{\rho c}{a(t)} \der{a(t)}{t} - \kappa \bigg(\frac{1}{\psi(x)}\der{^2 \psi(x)}{x^2} + \frac{1}{\phi(y)}\der{^2 \phi(y)}{y^2}\bigg) = 0.
\end{equation}
The three terms in the above equation are functions of the three independent variables $x, y, t$. So, in order for the above equation to be satisfied, each of the term must be constant. Consider a constant $\alpha > 0$. The three terms are then given as
\begin{equation}
\frac{\rho c}{a(t)} \der{a(t)}{t} = \alpha, \qquad \frac{1}{\psi(x)}\der{^2 \psi(x)}{x^2} =   \frac{1}{\phi(y)}\der{^2 \phi(y)}{y^2} = \frac{\alpha}{2 \kappa}.
\end{equation}
Upon simplifying, the following differential equations are obtained: 
\begin{subequations}
\begin{equation}
\der{a(t)}{t} - \lambda a(t) = 0,
\label{eqn:prob2_sub1}
\end{equation}
\begin{equation}
\der{^2 \psi(x)}{x^2} - \lambda_x \psi(x) = 0,
\label{eqn:prob2_sub2}
\end{equation}
\begin{equation}
\der{^2 \phi(y)}{y^2} - \lambda_y \phi(y) = 0,
\label{eqn:prob2_sub3}
\end{equation}
where $\lambda = \frac{\alpha}{\rho c}$, $\lambda_x  = \lambda_y = \frac{\alpha}{2 \kappa}$. \\
\end{subequations}
Thus it can be seen that for equation\eqref{eqn:prob2_soln} to be a solution to the PDE \eqref{eqn:prob2_model} when $u(x,y,t) = 0$, $a(t)$, $\psi(x)$ and $\phi(y)$ need to satisfy the equations \eqref{eqn:prob2_sub1}, \eqref{eqn:prob2_sub2} and \eqref{eqn:prob2_sub3} respectively.
%\begin{figure}[H]
%\begin{subfigure}{0.48\textwidth}
%\includegraphics[width=\linewidth]{22_21_ukappa_contour.eps}
%\end{subfigure}\hspace*{\fill}
%\begin{subfigure}{0.48\textwidth}
%\includegraphics[width=\linewidth]{22_21_ukappa_3d.eps}
%\end{subfigure}
%
%\medskip
%\begin{subfigure}{0.48\textwidth}
%\includegraphics[width=\linewidth]{42_41_ukappa_contour.eps}
%\end{subfigure}\hspace*{\fill}
%\begin{subfigure}{0.48\textwidth}
%\includegraphics[width=\linewidth]{42_41_ukappa_3d.eps}
%\end{subfigure}
%
%\medskip
%\begin{subfigure}{0.48\textwidth}
%\includegraphics[width=\linewidth]{82_81_ukappa_contour.eps}
%\end{subfigure}\hspace*{\fill}
%\begin{subfigure}{0.48\textwidth}
%\includegraphics[width=\linewidth]{82_81_ukappa_3d.eps}
%\end{subfigure}
%
%\caption{Comparison of the numerical (left) and exact (right) solutions for the 2D convection equation using the \textbf{unlimited $\kappa=1/3$ discretization in space}. The contours shown are for values of $c = 10^{-6} + n, \ n=0,\ldots, 9$.}
%\end{figure}

\section*{Problem 3}
\addcontentsline{toc}{section}{\protect\numberline{}Problem 3}

\section*{Problem 4}
\addcontentsline{toc}{section}{\protect\numberline{}Problem 4}
\bibliography{references} 
\bibliographystyle{ieeetr}
\pagebreak
\appendix
\end{document}